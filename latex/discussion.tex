\section{Properties of the model}

In many ways this model has its strength in its simplicity and ease of added complexity. 
By separating the two solvers, and having the possibility to run them both individually, changing one of them will not affect the other. 
Of course, added complexity often means added overhead by having to initialize more variables and parameters. 
This must be solved by adding functionality to the class in charge of running the two solvers and combining their results, which is considered a simple task. \\

The weakness of the model is mainly the geometry question. 
As has been mentioned before, the current version only supports quadratic meshes which of course limits the possible applications of the model. 
One possible workaround which has been suggested consists of changing the PDE solver to a finite element solver. 
However, this is associated with a large workload, and for some FEM software it will be nontrivial to map mesh points from the PDE solver to the RW solver.\\

Installation of the Armadillo linear algebra library is required for the BE scheme to work. 
This scheme is highly recommended to use both because it is the most well-tested solver, and because there is no stability criterion. 
Compiling the code without Armadillo installed requires some changing and exclusion of the code.