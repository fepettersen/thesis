\section{Properties of the model}



\section{Future work}
The developed model shows clear signs of being a first approach to the problem, and is in some ways a bit simple. 
Several improvements can be suggested to create a more realistic model both with respect to the diffusion processes on both length scales, and when it comes to the combination of the two models. 
This section will mention some of the improvements that can be done.

\subsection{PDE solver}
Although there is a well expressed limitation to the accuracy of the model determined by the stochastic term, an interesting extension to the software will be to introduce a finite element PDE Solver. 
Depending on which solver is implemented, this step will ensure that more complex (and realistic) properties of a diffusion process like nonlinearity can be introduced with relative simplicity. 
Another question is how this can be implemented in the lower-scale model..\\
A finite element solver will also, as discussed in section \ref{geometry}, vastly simplify the implementation of more realistic mesh geometries in more than 1D. \\
Experimenting with time derivative approximations that are better than first order could also be interesting seeing as a second order convergence was achieved by the similar model described in section \ref{simplified_test}. 
The actual benefit of introducing a better approximation to the time derivative must of course be tested.

\subsection{Lower scale models}
In section \ref{choosing_random_walk_algorithm} the argument that the Brownian motion model converges to the Gaussian distribution was given for choosing the simple RW model. 
The same argument can be made for most of the other possible models if they do not possess any special capabilities like drift or anisotropy. 
However, the argument is only valid in the verification phase when the number of walkers (or whatever) is large. 
For the actual simulations the number of walkers (or whatever) will typically be very small, and the central limit theorem does not apply. 
This opens the possibility of adding a variety of lower scale models, some of which will be mentioned below.

% \subsubsection{Molecular Dynamics}
% 
% \subsubsection{Direct Simulation Monte Carlo}
% 
% \subsubsection{Variations of Random Walk}

% \subsubsection{Dissipative Fluid Dynamics}