In this chapter we will take a closer look at random walks, both in general and the transition from the statistical view to partial differential equations. 
We will take a look at different algorithms to produce random walks, and discuss their pros and cons in light of this project.

\section{Introduction to random walks}
The most basic random walk is a walker on the x-axis which will take a step of a fixed length to the right with a probability $p$, or to the left with a probability $q=1-p$. 
Using (pseudo-) random numbers on a computer we can simulate the outturn of a random walk. 
For each step (of which there are N) we draw a random number, r, between 0 and 1 from some distribution (say a uniform one) which will be the probability. 
If $r\leq p$ the walker will take a step to the left, otherwise it will take a step to the left. 
After the N steps the walker will have taken R steps to the right, and $L = N-R$ steps to the left. 
The net displacement from the origin will be $S = R-L$. 

\subsection{Further discussion and analysis of the introduction}
If we do sufficiently many walks, the net displacement will vary from $S=+N$ to $S=-N$ representing all steps to the right and all steps to the left respectively. 
The probability of all steps beeing to the right is $P_N(N) = p^N$. 
Should one of the steps be to the left, and the rest to the right we will get a net displacement of $S = N-2$ with the probability $P_N(R = N-1) = Np^{N-1}q$. 
We can generalize this to finding the probability of a walk with a R steps to the right as 
\begin{equation}\label{bernoulli_distr}
 P_N(R) = {N\choose R}p^{R}q^{N-R}
\end{equation}
where ${N\choose R}=\frac{N!}{R!(N-R)!}$ is the number of walks which satisfy the net displacement in question, or the multiplicity of this walk in statistical mechanics terms. 
Equation \ref{bernoulli_distr} is the Bernoulli probability distribution, which is normalized.
\begin{align*}
 \sum\limits_{R=0}^N P_N(R) = (p+q)^N = 1^N = 1
\end{align*}
We can use this distribution to calculate various average properties of a walk consisting of N steps. 
For example, the average number of steps to the right is
\begin{align*}
 \langle R\rangle &=  \sum\limits_{R=0}^N RP_N(R) =  \sum\limits_{R=0}^N {N\choose R}Rp^Rq^{N-R} = \\
 p\frac{d}{dp} \sum\limits_{R=0}^N {N\choose R}p^Rq^{N-R} &= p\frac{d}{dp}(p+q)^N = Np(p+q)^{N-1} = Np
\end{align*}
From this we can also find the average value of the net displacement using $S = R-L = R-(N-R) = 2R-N$.
\begin{align*}
 \langle S\rangle = \langle2R\rangle -N = 2Np-N\underbrace{(p+q)}_{=1} = N(2p-p-q) = N(p-q)
\end{align*}
We notice that the average net displacement is greatly dependent on the probability of the walk and that any symmetric walk will have an expected net displacement of zero. 
In many cases we will be more interrested in the mean square displacement than the displacement itself. 
This can also be calculated rather straightforwardly. 
\begin{align*}
  \langle R^2\rangle =  \sum\limits_{R=0}^N R^2P_N(R) &=  \sum\limits_{R=0}^N {N\choose R}R^2p^Rq^{N-R} = \\
 \left(p\frac{d}{dp}\right)^2 \sum\limits_{R=0}^N {N\choose R}p^Rq^{N-R} &= \left(p\frac{d}{dp}\right)^2(p+q)^N \\
 = Np(p+q)^{N-1} +p^2N(N-1)(p+q)^{N-2} &= (Np)^2 +Np(1-p) = (Np)^2 +Npq
\end{align*}
Like before, the average nett displacement is given as $S^2 = (2R-N)^2$ and we obtain
\begin{align*}
 \langle S^2\rangle = 4\langle R^2\langle -4N\langle R\rangle + N^2 &= 4((Np)^2 +Npq) -4N^2p + N^2\\
 = N^2(4p^2 -4p +1) +4Npq &= N^2(2p-1)^2 +4Npq = N^2(p-q)^2 +4Npq
\end{align*}
which for the 1D symmetric walk gives $\langle S^2\rangle =N$

