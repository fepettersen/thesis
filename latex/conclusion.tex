This (last) chapter contains the conclusion and sums up what has been done, what went wrong and finally suggests some future improvements and extensions that would be interesting to implement.

\section{Workflow}

In summary this thesis has two parts; the implementation and testing of both the PDE and RW models and the software which combines them, and the implementation and simulations of the diffusion of PKC$\gamma$ into spines - problem. 
The former has without doubt been the most time consuming, mostly because of a bug which made the results appear correct without them being so, but also because of several other minor bugs which resulted in redoing most of the verifications several times.\\

Quite a lot of time went into assembly of the mass-matrix for the BE scheme in 2D. 
This was also because of a small bug causing the boundary conditions on one boundary to rely on the wrong parts of the previous time step. 
Assembling a mass matrix for a 3D BE scheme will probably involve some messy calculations, but is not considered difficult (since this has been done on paper already).


\section{The model}

As was mentioned in section \ref{results:validity_of_the_model} the analysis suggests that the developed model is stable and gives good results. 
The important detail which makes the model work is to do the RW ``integration'' first, pass the result as input to the PDE model and then solve the PDE by a method of choice. 
It is unclear whether a finite element method will give as good results, but highly probable seeing as it is the fundamental property of a diffusion process which ensures this. Namely that a diffusion process will dampen fast fluctuations more efficiently than slow fluctuations.

\section{Future work}
The developed model shows clear signs of being a first approach to the problem, and is in some ways a bit simple. 
Several improvements can be suggested to create a more realistic model both with respect to the diffusion processes on both length scales, and when it comes to the combination of the two models. 
This section will mention some of the improvements that can be done.

\subsection{PDE solver}
Although there is a well expressed limitation to the accuracy of the model determined by the stochastic term, an interesting extension to the software will be to introduce a finite element PDE Solver. 
Depending on which solver is implemented, this step will ensure that more complex (and realistic) properties of a diffusion process like nonlinearity can be introduced with relative simplicity. 
Another question is how this can be implemented in the lower-scale model..\\
A finite element solver will also, as discussed in section \ref{geometry}, vastly simplify the implementation of more realistic mesh geometries in more than 1D. \\
Experimenting with time derivative approximations that are better than first order could also be interesting seeing as a second order convergence was achieved by the similar model described in section \ref{simplified_test}. 
The actual benefit of introducing a better approximation to the time derivative must of course be tested.

\subsection{Lower scale models}
In section \ref{choosing_random_walk_algorithm} the argument that the Brownian motion model converges to the Gaussian distribution was given for choosing the simple RW model. 
The same argument can be made for most of the other possible models if they do not possess any special capabilities like drift or anisotropy. 
However, the argument is only valid in the verification phase when the number of walkers (or whatever) is large. 
For the actual simulations the number of walkers (or whatever) will typically be very small, and the central limit theorem does not apply. 
This opens the possibility of adding a variety of lower scale models, some of which will be mentioned below.

% \subsubsection{Molecular Dynamics}
% 
% \subsubsection{Direct Simulation Monte Carlo}
% 
\subsubsection{Variations of Random Walk}
Although the current RW implementation supports some added complexity like anisotropy and drift, there is always the possibility to make the algorithm more complex. 
Of course, there is not much reason to do this without an actual physical problem which results in some more complex RW algorithm, but finding these applications should not be to hard. 
For example, some attraction/drift term could be added to simulate Coulomb-attraction.

% \subsubsection{Dissipative Fluid Dynamics}