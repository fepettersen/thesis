\documentclass{beamer}

\usepackage[utf8]{inputenc}
\usepackage{default}
\usepackage{graphicx}
\usepackage{subfig}
\usepackage{rotating}
\usepackage{multimedia}

\usepackage[]{biblatex}
% \bibliography{refs/refs.bib}

% \pdfminorversion=5

%-----------------------------------------------------%
\renewcommand{\d}{\partial}
%-----------------------------------------------------%

\newcommand{\subfigure}{\subfloat}

\usetheme{Warsaw}
\title[Presentation - Kongsberg Defence \& Aerospace]{Diffusion processes in the brain}
\author{Fredrik E. Pettersen}
\date{\today}



\begin{document}

\begin{frame}
\titlepage
\end{frame}



\begin{frame}
 \frametitle{Contents}
 \tableofcontents[hideallsubsections]
\end{frame}

\section{Introduction to neuroscience}
\begin{frame}
 \frametitle{The Central Nervous System}
 \begin{columns}
\column{0.48\textwidth}
 \begin{itemize}
  \item All invertebrates except sponges and radially symmetric animals have one.
  \item Consist of spinal cord and brain in vertebrates.
  \item Tasked with gathering and processing information.
 \end{itemize}
  \column{0.48\textwidth}
  \begin{figure}[H]
  \centering
  \includegraphics[scale=0.27]{figures/CNS.jpg}
  \caption{Human CNS}
  \end{figure}
 \end{columns}
\end{frame}



\begin{frame}
 \frametitle{Some words about the brain}
 \begin{columns}
\column{0.48\textwidth}
 \begin{itemize}
  \item Labeled the most complex object in the universe.
  \item $\sim200$ billion neurons with $\sim125$ trillion connections in neocortex alone.
  \item Different parts associated with different tasks.
  \item Many underlying processes are very inefficient.
 \end{itemize}
  \column{0.48\textwidth}
  \begin{figure}[H]
  \centering
  \includegraphics[scale=0.27]{figures/human_brain.jpg}
  \caption{Human brain with labels}
  \end{figure}
 \end{columns} 
\end{frame}


\begin{frame}
 \frametitle{Cells in the brain}
 \begin{columns}
\column{2.0in} Neurons:\\
\begin{itemize}
 \item Signal processing
\end{itemize}
\begin{figure}[H]
 \centering
 \includegraphics[width=\textwidth]{figures/neuron.jpg}
\end{figure}

\column{2.0in} Neuroglia:\\
\begin{itemize}
 \item Janitorial tasks
\end{itemize}
\begin{figure}[H]
 \centering
 \includegraphics[width=\textwidth]{figures/neuroglia.jpg}
\end{figure}
 \end{columns}
\end{frame}



\section{Diffusion}

\begin{frame}
 \frametitle{Normal diffusion}
 \begin{itemize}
  \item Process of net movement due to a difference in concentration.
  \item Formulated in 1855 by Adolf Fick in the modern way.
  \item Widely used across many diciplines like social studies, economics and biology.
 \end{itemize}

 \begin{equation*}
  \frac{\d C}{\d t} = D\nabla^2C
 \end{equation*}
\begin{figure}[H]
 \centering
 \includegraphics[width=\textwidth]{figures/Diffusion.png}
\end{figure}

\end{frame}



\begin{frame}
 \frametitle{Random walks}
 \begin{columns}
  \column{0.48\textwidth}
  \begin{itemize}
   \item Also widely used in many diciplines.
   \item ``Endless'' possibilities for added complexity.
   \item Conceptually not that difficult.
   \item Recreates diffusion
  \end{itemize}
\column{0.48\textwidth}
\begin{figure}[H]
\centering
\includegraphics[width=\textwidth]{figures/RW.jpg}
 \end{figure}

 \end{columns}
\end{frame}


\section{Diffusion in the brain}
\begin{frame}
 \frametitle{Diffusion across synapses}
 \begin{columns}
\column{0.48\textwidth}
 \begin{itemize}
 \item Two types of synapses connect neurons - electrical and chemical.
 \item Action potentials triggers release of neurotransmitter into synaptic cleft.
 \item Receiving end passes input on to cell body.
 \item Diffusion across synaptic cleft takes $\sim\mu$s or less.
 \end{itemize}
\column{0.48\textwidth}
\begin{figure}[H]
 \centering
 \includegraphics[width=\textwidth]{figures/synapse.jpg}
 \caption{Chemical synapse with dendritic spine.}
\end{figure}
 \end{columns} 
\end{frame}

\begin{frame}
 \frametitle{PKC$\gamma$ diffusion into spines}
 \begin{columns}
  \column{2.0in}
  \begin{itemize}
   \item PKC$\gamma$ is an enzyme associated with learning.
   \item Released from cell body and diffuses through dendrite into spines.
   \item Very low concentrations could require multi scale modeling.
  \end{itemize}
\column{2.0in}
\begin{figure}[H]
\centering
% \includegraphics[width=\textwidth]{octahedra.png}
\end{figure}

 \end{columns}

\end{frame}



\begin{frame}
\begin{center}
 \textbf{Thank you!}
\end{center}
% \printbibliography
\end{frame}





% \section{Some interesting effects}
\begin{frame}
 \frametitle{Firing in auditory nervous system}
\end{frame}


\begin{frame}
\frametitle{Cells with specific tasks}

\end{frame}

\begin{frame}
\frametitle{Visual cortex}
% \movie[autostart,externalviewer=vlc,start=60]{}{Visual_cortex.mp4}
\end{frame}

\begin{frame}
 \frametitle{Lobes}
 Frontal lobe is associated with short term memory, social intelligence, reward, attention and planning\\
 Pareital lobe is in charge of integrating sensory information like spatial orientation\\
 The temporal lobes are involved in the retention of visual memories, comprehending language, storing new memories, emotion, and deriving meaning.\\
 The Occipital lobe is mainly in charge of processing visual information.
\end{frame}


\begin{frame}
\frametitle{From RW to diffusion}
A large number, N, of walkers can be described by their concentration $C(x,t) = NP(x,t)$.
% If we have a large number, N, of walkers, their concentration will be $C(x,t) = NP(x,t)$. 
The concentration is conserved, so any amount that flows out of an area must reflect as a decrease in concentration. 
This is expressed by the flow of concentration
\begin{equation}
 \frac{\d C}{\d t} -\nabla\cdot\vec{J} = S
\end{equation}
where $\vec{J}$ is the flow vector and S is a source term which for now is zero.
Through Fick's first law the diffusive flux is related to the concentration gradient $\vec{J} = -D\nabla C$. 
Inserting this gives
\begin{equation}\label{simple_diffusion_equation}
 \frac{\d C}{\d t} = \nabla\cdot \left(D\cdot\nabla C\right)
\end{equation}
\end{frame}

\begin{frame}
\frametitle{Applications of RW}
\begin{itemize}
 \item GPS map-navigation
 \item Percolation theory (for flow simulations)
 \item Estimate size of internet
\end{itemize}

\end{frame}

\end{document}