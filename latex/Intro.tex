This thesis is an attempt at modeling diffusion processes in which part of the process takes place on a length scale so small that the continuum approximation becomes invalid. 
In this part we will therefore try and introduce another model of the diffusion process in the hope that this will give us something extra. 
It is the hope of Hans Petter that this thesis will be an introduction to a new research project to understand mesoscale physics in collaboration with Gaute Einevoll at UMB.

The very first approach was to simply try the problem on a bit. That is to try and substitute some small part of the mesh in a Finite Difference Diffusion solver (Froward Euler scheme) with a stochastic diffusion solver. A random walk method was implemented on part of the mesh to take over the equation-solving. This was done in 1 and 2 spatial dimensions with the aim of finding potential difficulties so that we can further investigate them. \\
Upon switching length-scales a fundamental question arises almost immediately; what is the continuum limit? In our case this question takes a slightly different, and possibly more answerable form; what is the conversion rate between the continuum model and the microscopic model, and by extension, what does a walker correspond to?
The first instinct of this candidate was to just try some conversion rate (say some value corresponds to some number of walkers), and this was implemented in both 1 and 2 dimensions.

\section{Background}
\emph{This is very much a first draft, and only my thoughts around what I understand as the basis of the thesis.}\\
Though the scope of this thesis might seem a bit \emph{sought}, it is in fact a real world concern from the computational neuroscience group at UMB. 
Their work contains network simulations of neurons, which they are trying to tweak with experimental data. 
The experimental data come from measurements of voltage levels in the Extra Cellular Space (ECS). 
The ECS is, essentially, the space which separates neurons and has an immensely complicated geometry as we can see from figure \ref{ECS_picture}. 
The width of the ECS varies, but is in general $\sim1\mu$m. 
Parts of the ECS is, however, much narrower. In the ECS there are a number of diffusion governed transport mechanisms which are vital for the function of the neurons. 
In fact, there are examples of snake venom which have a shrinking effect on the ECS, and it is thought that this causes large scale neuronal death very quickly. 
For the researchers at UMB, however, the ECS is of importance because the diffusion tensor in the ECS is related to the conductivity tensor through equation \ref{conductivity_diffusion_eq}, which in turn lets them teak parameters in their network models. \\
\begin{equation}\label{einstein_viscosity}
D = \frac{k_B T}{6\pi \eta r},
\end{equation}
Though there are several reasons to study diffusion processes in the ECS, this project has a specific goal in mind. 
The Einstein relation, eq. (\ref{einstein_viscosity}), relates the diffusion constant to the viscosity of the medium in which the diffusion is taking place. 
From the definition of viscosity, $\mu$, we have 
\begin{equation}
v_d = \mu F 
\end{equation}
where $F = qE$ is the standard electrical force acting on a charged particle. 
We can also define the current from the drift velocity of the particles as 
\begin{equation}
J = cqv_d = \sigma E 
\end{equation}
where $\sigma = c\mu q^2$ is the electrical conductance, in this case, of the ECS. 
Inserting this in the Einstein relation, eq. (\ref{einstein_viscosity}), lets us express the conductivity in terms of the diffusion constant as 
\begin{equation}\label{conductivity_diffusion_eq}
\sigma = \frac{cq}{k_B T}D 
\end{equation}
Equation (\ref{conductivity_diffusion_eq}) is trivially generalizable to tensor notation, where the conductance and diffusion ``constant'' are both 2. order tensors.\\

\begin{equation}\label{stochastic_einstein}
\langle r^2\rangle = 2dDt 
\end{equation}

The fundamental problem here is that, while diffusion in its own is a truly multi-scale process, we have no way of knowing for sure that the continuum models, and all they bring with them, are correct for this type of geometry. 
In particular, the Einstein relation \ref{einstein_viscosity} has, to my knowledge, only been derived for diffusion in a homogeneous media. 
The definition of a homogeneous media includes a mean free path close to infinity, which we will have trouble arguing for the existence of in the ECS. 
On the other hand, I know that another Einstein relation \ref{stochastic_einstein} is widely used  in other fields of physics where the media in question is hardly homogeneous. 
In molecular dynamics simulations, the two mentioned Einstein relations are used to measure the viscosity of fluids in nano-porous materials, and with success as far as I know.

\section{A short introduction to mathematical neuroscience}

Neuroscience is the scientific study of the nervous system, but in the traditional sense it focuses very much on what parts of the brain are responsible for what. 
Mathematical neruroscience is more focused on the physics and chemistry involved in the different parts of the brain and nervous system. 
An example of the power of this approach is the classical work done by Hodgkin and Huxley in 1952, earning them the Nobel price in Physiology or medicine in 1963. 
Through four non-linear coupled differential equations they were able to predict the propagation speed of signals along the squid giant axon to a quite high precision. \\

% Copied from the project

The human brain consists of two types of cells; the neurons and neuroglia. 
Neurons are tasked with signal processing and transport, while the glia are thought to have more janitorial tasks. 
The neurons are bathed in a salt solution that is mainly $Na^+$ and $Cl^-$. 
Inside the neurons, a highly regulated salt solution of mainly $K^+$ sets up a potential difference relative to the outside of the cell of approximately $-65$mV.
The neurons are in constant communication with each other through action potentials, which are disturbances in the membrane potentials of neurons. 
These action potentials are generated in the body of the cell, called the soma, from where they propagate down the axon without loss of amplitude. 
This is achieved by constantly amplifying the signal using ion pumps (see the Hodgin-Huxley model of the action potential \cite{sterratt2011principles}).
After propagating down the axon, the action potential reaches a synapse which is a gate to another neuron. 
If the action potential is of significant strength, vesicles carrying neurotransmitters merge with the synapse membrane, letting the neurotransmitters diffuse to the dendrite of the other neuron. 
If enough neurotransmitters reach the post-synaptic side, the signal continues propagating to the soma of this neuron, and the entire process starts over again. \\
The interest of this project lies, mainly, in the diffusion processes that take place in the space between these types of cells, the so-called extracellular space (ECS). 
This is a narrow space ($\sim 10-100$ nm \cite{nicholson2001diffusion}) with a highly complicated geometry (Figure \ref{ECS}). 
Surprisingly, the ECS adds up to $20\%$ of the total brain volume. 
We can understand this by realizing that every part of a cell must be separated from another cell by the ECS. 
Since the cells consists of axons and dendrites which can be viewed as (somewhat) fractal, we see that this means separating a vast amount of surface area from other surface areas.

The ECS is thought to support the diffusion of oxygen and nutrients to the neurons and glia, and diffusion of carbon dioxide and other waste from these cells through the blood - brain barrier and into the bloodflow. 

% \section{Dendritic spines}


