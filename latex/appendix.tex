\section{Appendix}

In this appendix some more tedious and rather boring, but no less important calculations can be found. 
We will also list some algorithms that are important, but not quite in the scope of this thesis.

\subsection{Alternate Direction Implicit scheme}
Starting with the discretization in equation \ref{} we can find the entire linear problem for a $3\times3$ grid with the no-flux Neumann boundary conditions
\begin{equation*}
 \frac{\d u}{\d n}|_{\d\Omega} = 0
\end{equation*}
which gives us $\frac{u^{n+1/2}-u^n}{\Delta t/2} = 0 \implies u^{n+1/2}=u^n$.
The general discretization written out gives us
\begin{align*}
 -\alpha u_{i+1,j}^{n+1/2} +(1+2\alpha)u_{i,j}^{n+1/2} -\alpha u_{i-1,j}^{n+1/2} &= -\beta u_{i,j+1}^{n}+(1-2\beta)u_{i,j}^{n} +\beta u_{i,j-1}^{n} \\
 -\beta u_{i,j+1}^{n+1} +(1+2\beta)u_{i,j}^{n+1} -\beta u_{i,j-1}^{n+1} &=\alpha u_{i+1,j}^{n+1/2} +(1-2\alpha)u_{i,j}^{n+1/2} +\alpha u_{i-1,j}^{n+1/2}
\end{align*}
Where 
\begin{align*}
 \alpha = \frac{D\Delta t}{2(\Delta x)^2}\text{,}\; \beta = \frac{D\Delta t}{2(\Delta y)^2}
\end{align*}
Solving the two equations above on the mentioned $3\times3$ grid gives us 18 equations which in turn can be written as two tridiagonal linear equations if we rewrite the unknown matrices $u^{n+1/2}$ and $u^n$ as column vectors (see eqs. \ref{ADI_vectors_step1} and \ref{ADI_vectors_step2}). 

\begin{align}
  \left(\begin{array}{c c c c c c c c c}
        1+2\alpha & -2\alpha &0 &0 &0 &0 &0 &0 &0\\
        -\alpha & 1+2\alpha & -\alpha &0 &0 &0 &0 &0 &0\\
        0& -2\alpha & 1+2\alpha & 0 & 0 & 0 &0&0&0\\
        0& 0&0 & 1+2\alpha & -2\alpha & 0 & 0 &0&0\\
        0& 0&0&-\alpha & 1+2\alpha & -\alpha & 0 & 0 &0\\
        0& 0&0&0&-2\alpha & 1+2\alpha & 0 & 0 &0\\
        0& 0 &0 &0 &0&0 & 1+2\alpha & -2\alpha&0\\
        0& 0 &0 &0 &0&0&-\alpha & 1+2\alpha &-\alpha\\
         0&0 &0 &0&0 &0&0&-2\alpha & 1+2\alpha
       \end{array}\right)\mathbf{u}^{n+1/2} = \mathbf{f}
\end{align}
and
\begin{align}
  \left(\begin{array}{c c c c c c c c c}
        1+2\beta & -2\beta &0 &0 &0 &0 &0 &0 &0\\
        -\beta & 1+2\beta & -\beta &0 &0 &0 &0 &0 &0\\
        0& -2\beta & 1+2\beta & 0 & 0 & 0 &0&0&0\\
        0& 0&0 & 1+2\beta & -2\beta & 0 & 0 &0&0\\
        0& 0&0&-\beta & 1+2\beta & -\beta & 0 & 0 &0\\
        0& 0&0&0&-2\beta & 1+2\beta & 0 & 0 &0\\
        0& 0 &0 &0 &0&0 & 1+2\beta & -2\beta&0\\
        0& 0 &0 &0 &0&0&-\beta & 1+2\beta &-\beta\\
         0&0 &0 &0&0 &0&0&-2\beta & 1+2\beta
       \end{array}\right)\mathbf{u}^{n+1} = \mathbf{f}_2
\end{align}
The vectors $\mathbf{u}^{n+1/2}$, $\mathbf{f}$, $\mathbf{u}^{n+1}$ and $\mathbf{f}_2$ are shown in eqs. \ref{ADI_vectors_step1} and \ref{ADI_vectors_step2}
\begin{align}
  \mathbf{u}^{n+1/2} = 
  \left(\begin{array}{c}
        u^{n+1/2}_{0,0}\\u^{n+1/2}_{0,1}\\u^{n+1/2}_{0,2}\\
        u^{n+1/2}_{1,0}\\u^{n+1/2}_{1,1}\\u^{n+1/2}_{1,2}\\
        u^{n+1/2}_{2,0}\\u^{n+1/2}_{2,1}\\u^{n+1/2}_{2,2}
       \end{array}\right)\text{,}\;
 \mathbf{f} = 
  \left(\begin{array}{c}
        (1-2\beta)u^{n}_{0,0}+2\beta u^{n}_{0,1}\\ \beta u^{n}_{0,0}+ (1-2\beta)u^{n}_{0,1}+\beta u^{n}_{0,2} \\(1-2\beta)u^{n}_{0,2}+2\beta u^{n}_{0,1}\\
        (1-2\beta)u^{n}_{1,0}+2\beta u^{n}_{1,1}\\ \beta u^{n}_{1,0}+ (1-2\beta)u^{n}_{1,1}+\beta u^{n}_{1,2}\\ (1-2\beta)u^{n}_{1,2}+2\beta u^{n}_{1,1}\\
        (1-2\beta)u^{n}_{2,0}+2\beta u^{n}_{2,1}\\ \beta u^{n}_{2,0}+ (1-2\beta)u^{n}_{2,1}+\beta u^{n}_{2,2}\\(1-2\beta)u^{n}_{2,2}+2\beta u^{n}_{2,1}
       \end{array}\right)\label{ADI_vectors_step1} \\
 \mathbf{u}^{n+1} = 
  \left(\begin{array}{c}
        u^{n+1}_{0,0}\\u^{n+1}_{0,1}\\u^{n+1}_{0,2}\\
        u^{n+1}_{1,0}\\u^{n+1}_{1,1}\\u^{n+1}_{1,2}\\
        u^{n+1}_{2,0}\\u^{n+1}_{2,1}\\u^{n+1}_{2,2}
       \end{array}\right)\text{,}\;
 \mathbf{f}_2 = 
  \left(\begin{array}{c}
        (1-2\alpha)u^{n+1/2}_{0,0}+2\alpha u^{n+1/2}_{1,0}\\(1-2\alpha)u^{n+1/2}_{0,1}+2\alpha u^{n+1/2}_{1,1}\\(1-2\alpha)u^{n+1/2}_{0,2}+2\alpha u^{n+1/2}_{1,2}\\
         \alpha u^{n+1/2}_{2,0}+ (1-2\alpha)u^{n+1/2}_{1,0}+\alpha u^{n+1/2}_{0,0} \\ \alpha u^{n+1/2}_{2,1}+ (1-2\alpha)u^{n+1/2}_{1,1}+\alpha u^{n+1/2}_{0,1}\\
         \alpha u^{n+1/2}_{2,2}+ (1-2\alpha)u^{n+1/2}_{1,2}+\alpha u^{n+1/2}_{0,2}\\(1-2\alpha)u^{n+1/2}_{1,0}+2\alpha u^{n+1/2}_{2,0}\\
	  (1-2\alpha)u^{n+1/2}_{1,1}+2\alpha u^{n}_{2,1}\\ (1-2\alpha)u^{n+1/2}_{1,2}+2\alpha u^{n+1/2}_{2,2}
       \end{array}\right)\label{ADI_vectors_step2}
\end{align}

\subsection{Backward Euler scheme in 2D}
Using the BE discretization on the simple 2D diffusion equation will yield the general scheme in equation \ref{general_scheme_BE2D}.
\begin{equation}\label{general_scheme_BE2D}
 u^{n}_{i,j} = \underbrace{\frac{-D\Delta t}{\Delta x^2}}_{\alpha}\left(u^{n+1}_{i+1,j}+u^{n+1}_{i-1,j}\right) +
 \underbrace{\left(1+\frac{2D\Delta t}{\Delta x^2} +\frac{2D\Delta t}{\Delta y^2}\right)}_{\gamma}u^{n+1}_{i,j} 
 \underbrace{-\frac{2D\Delta t}{\Delta y^2}}_{\beta}\left(u^{n+1}_{i,j+1}+u^{n+1}_{i,j-1}\right)
\end{equation}
This can, again, be written as a linear problem where the vectors are simply the matrices $u^n$ and $u^{n+1}$ written as column vectors. 
The matrix is written out for a $3\times3$ grid with no-flux Neumann boundary conditions in equation \ref{linear_system_BE2D}. 
We see that it is a five-band diagonal matrix, and so the tridiagonal solver cannot be used in this case. It is fully possible to use for example a Gaussian elimination in order to solve this equation, but it will require $\frac{2}{3}\mathcal{O}(n^3)$ operations per time step, where n is the size of the matrix (in this case $n=9$). Another way to solve this equation, and by extension use the BE scheme, is to use some form of sparse LU decomposition.
\begin{align}\label{linear_system_BE2D}
  \left(\begin{array}{c c c c c c c c c}
        \gamma & -2\beta &0 &-2\alpha &0 &0 &0 &0 &0\\
        -\beta & \gamma & -\beta &0 &-2\alpha &0 &0 &0 &0\\
        0&-2\beta & \gamma & 0 & 0 & -2\alpha &0&0&0\\
        -\alpha& 0&0 & \gamma & -2\beta & 0 & -\alpha &0&0\\
        0& -\alpha&0&-\beta & \gamma & -\beta & 0 & -\alpha &0\\
        0& 0& -\alpha&0&-2\beta & \gamma & 0 & 0 &-\alpha\\
        0& 0 &0 &-2\alpha &0&0 & \gamma & -2\beta&0\\
        0& 0 &0 &0 &-2\alpha&0&-\beta & \gamma &-\beta\\
         0&0 &0 &0&0 &-2\alpha&0&-2\beta & \gamma
       \end{array}\right)\mathbf{u}^{n} = \mathbf{u}^{n+1}
\end{align}