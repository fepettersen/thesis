\section{Validity of the model}
As we saw in the introduction to the analysis chapter it is possible to introduce enough walkers so that there is no mathematical difference between the combined model and the PDE-model. 
For all practical purposes, however we will not use that many walkers, because it is not the reason we introduced them in the first place. 
The simulations done in the analysis-chapter are largely done for verification purposes (see chapter \ref{debugging} in the appendix).
In other words, our model converges to the continuum model in the limit of sufficient walkers.

% The results of the testing done in the Analysis-chapter, particularly the convergence tests based on varying the conversion-factor, Hc, to exceed the expected numerical error arising from the scheme itself based on equation \ref{}, suggest that the implementation of our model is correct. 
% The mathematics overlap, as we have seen in chapter \ref{}, and if we introduce enough walkers there seems to be no difference between adding an area on the mesh where random walks solve the equation, and not doing so. 

\section{Diffusion times into spines}

Craske et.al. suggest that the neck of spines act as diffusion barriers which slow down, but don't completely stop the diffusion of PKC$\gamma$ into spines. 
The function of this barrier is a bit unclear, but the presence of it is undisputed. 
In their measurements they found a delay of around $5-10$ seconds from elevated concentration levels in the dendrite until a similarly elevated concentration level occurred in the spine. 
Using parameter values which resemble the values found in actual (rodent) neurons and neurites in the developed software, the observed delay-times have been recreated. 
Seeing as there are no additional complexities added to the random walk model we can assume that the spine neck does in fact function as a diffusion barrier.

Through the simulations it became apparent that there must be some sort of limiting factor which limits the number of PKC$\gamma$ particles that are let into the spine. 
In real life this is achieved by a concentration gradient which tends to zero (or negative values) meaning that no particles will diffuse into the spine after it is ``filled'' up. 
A random walker will not feel this concentration gradient unless it is explicitly told so. 
The alternative solution then, is to reduce the probability for particles to diffuse into a spine for each particle that get caught in the spine head by some reasonable factor. 