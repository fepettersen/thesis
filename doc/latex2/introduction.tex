The diffusion equation governs a large number of physical processes on many different length scales. 
Both heat diffusion through a wall and the Brownian motion of dust particles can be described by the diffusion equation. 
In some diffusion processes a large number of particles will diffuse from a large volume into a small volume in such a way that only very few particles enter the smaller volume. 
An example of this in the diffusion of the enzyme Protein Kinase C$\gamma$ (PKC$\gamma$\nomenclature{PKC$\gamma$}{Protein Kinase C$\gamma$}) from the body of a brain cell via a dendrite of large volume into dendritic spines. 
Dendritic spines are the receiving end of a connection between two brain cells, and PKC$\gamma$ is associated with long term reinforcement of such a connection resulting in learning or associative memory storage. 
The increased concentration of PKC$\gamma$ in a dendritic spine is measured to around $5\frac{\text{nMol}}{\text{L}}$ which translates to $1-2$ PKC$\gamma$ enzymes in the spine. 
Seeing as the diffusion equation is derived from a continuous concentration distribution, using it on the enzymes in the spine seems questionable. \\

This thesis will look at hybrid diffusion processes where part of the process has very few particles and other parts have enough particles to be described by a continuous diffusion equation. 
The part with few particles will have a higher resolution in both space and time, and will be modeled by a stochastic process. 
A large emphasis has been put on verification of all parts of the hybrid model. 
Both individually and for the combined, hybrid solver to verify the average behavior of the system.
