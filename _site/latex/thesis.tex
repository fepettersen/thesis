\documentclass[twoside,english]{uiofysmaster}
\usepackage[utf8]{inputenc}
\usepackage[T1]{fontenc}
\usepackage[english]{babel}
\usepackage{epsfig}
\usepackage{graphicx}
\usepackage{caption}
\usepackage{subcaption}
\usepackage{amsfonts, amssymb, amsmath}
\usepackage{listings}
\usepackage{float}
\usepackage[top=2cm, bottom=2cm, left=2cm, right=2cm]{geometry}

% \usepackage{fontspec}
% \newfontfamily\listingsfont[Scale=0.85]{Droid Sans Mono}
% \lstset {
%     basicstyle=\footnotesize\listingsfont,
%     keywordstyle=\color{listingskeywordcolor}\footnotesize\listingsfont,
%     stringstyle=\color{listingsstringcolor}\footnotesize\listingsfont,
%     commentstyle=\color{listingscommentcolor}\footnotesize\listingsfont,
%     numberstyle=\color{listingsnumbercolor}\footnotesize\listingsfont,
%     identifierstyle=\footnotesize\listingsfont,
% }

%opening
\author{Fredrik E Pettersen\\ f.e.pettersen@fys.uio.no}
\title{\uppercase{Multiscale modeling of diffusion processes in the brain}}
\date{June 2013}
\begin{document}

%\bibliography{references}

\maketitle

\begin{abstract}
This is an abstract text.
\end{abstract}

\begin{dedication}
  To someone
  \\\vspace{12pt}
  This is a dedication to my cat.
\end{dedication}

\begin{acknowledgements}
  I acknowledge my acknowledgements.
\end{acknowledgements}

\tableofcontents
\clearpage
\listoffigures
\clearpage
\listoftables

\chapter{The beginning is here}

Start your chapter by writing something smart. Then go get coffee.\\
The very first approach was to simply try the problem on a bit. That is to try and substitute some small part of the mesh in a Finite Difference Diffusion solver (Froward Euler scheme) with a stochastic diffusion solver. A random walk method was implemented on part of the mesh to take over the equation-solving. This was done in 1 and 2 spatial dimensions with the aim of finding potential difficulties so that we can further investigate them. \\
Upon switching length-scales a fundamental question arises almost immediately; what is the contiuum limit? In our case this question takes a slightly different, and possibly more answearble form; what is the conversion rate beween the continuum model and the microscopic model, and by extension, what does a walker correspond to?
The first instinct of this candidate was to just try some conversion rate (say some value corresponds to som enumber of walkers), and this was implemented in both 1 and 2 dimensions.
\end{document}

\end{document}

